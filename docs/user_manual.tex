\documentclass[a4paper,12pt]{article}

% Базовые пакеты
\usepackage{geometry}
\usepackage{graphicx}
\usepackage{enumitem}
\usepackage{xcolor}
\usepackage{fancyhdr}
\usepackage{tcolorbox}

% Настройка геометрии страницы и колонтитулов
\geometry{a4paper, margin=2.5cm}
\setlength{\headheight}{15pt} % Увеличиваем высоту колонтитула для fancyhdr

% Настройки для русского языка и шрифтов с поддержкой кириллицы
\usepackage{fontspec}
\usepackage{polyglossia}
\setdefaultlanguage{russian}

% Настройка шрифтов для кириллицы
\setmainfont{Times New Roman}
\setsansfont{Arial}
\newfontfamily\cyrillicfont{Times New Roman}
\newfontfamily\cyrillicfontsf{Arial}

% Важно: установка моноширинного шрифта с поддержкой кириллицы
% Используем шрифты, встроенные в macOS
\setmonofont{Menlo}[Scale=0.9]
\newfontfamily\cyrillicfonttt{Menlo}[Scale=0.9]

% Запасной вариант, если Menlo недоступен
% \setmonofont{Courier New}[Scale=0.9]
% \newfontfamily\cyrillicfonttt{Courier New}[Scale=0.9]

% Настройка гиперссылок - должна быть после fontspec и polyglossia
\usepackage{hyperref}
\hypersetup{colorlinks=true, linkcolor=blue, urlcolor=blue}

\title{\textbf{Руководство пользователя}\\
{\huge StoreKeeper}\\
Система сбора и анализа выручки магазинов}
\author{Документация версии 1.0}
\date{\today}

\pagestyle{fancy}
\fancyhead[L]{StoreKeeper}
\fancyhead[R]{Руководство пользователя}
\fancyfoot[C]{\thepage}

\begin{document}

\maketitle
\thispagestyle{empty}
\tableofcontents
\newpage

\section{Введение}

StoreKeeper — это Telegram-бот, разработанный для сбора, учета и анализа выручки в сети магазинов. Система предоставляет различные функциональные возможности в зависимости от роли пользователя в системе (администратор или менеджер).

\subsection{Основные возможности}
\begin{itemize}
    \item Сбор и ввод данных о ежедневной выручке магазинов
    \item Отслеживание выполнения плана продаж
    \item Генерация визуальных отчетов в формате матрешек
    \item Экспорт данных в Excel
    \item Управление магазинами и персоналом
\end{itemize}

\section{Начало работы}

\subsection{Вход в систему}
Для начала работы с ботом отправьте команду \texttt{/start} в чат с ботом StoreKeeper.

\begin{tcolorbox}[colback=gray!10, title=Авторизация]
\textbf{Для администратора:} Авторизация происходит автоматически по ID чата.

\textbf{Для менеджера:} Введите свое имя и фамилию через пробел в ответ на запрос бота.
\end{tcolorbox}

\section{Руководство для менеджеров}

Менеджеры имеют доступ к функциям ввода данных о выручке своего магазина и просмотра статистики выполнения плана.

\subsection{Основное меню}
После успешной авторизации вам будет доступно меню с командами:
\begin{itemize}
    \item \texttt{/revenue} — ввод выручки за определенную дату
    \item \texttt{/status} — проверка статуса выполнения плана
    \item \texttt{/help} — вызов справки с командами
\end{itemize}

\subsection{Ввод данных о выручке}
\begin{enumerate}
    \item Отправьте команду \texttt{/revenue}
    \item Выберите дату из предложенного календаря (сегодня или одна из предыдущих 7 дней)
    \item Если вы привязаны к магазину, система автоматически определит ваш магазин. В противном случае выберите магазин из списка.
    \item Введите сумму выручки числом (используйте точку или запятую для разделения десятичных долей).
    \item Получите подтверждение о сохранении данных.
\end{enumerate}

\begin{tcolorbox}[colback=blue!5, title=Пример ввода выручки]
\begin{verbatim}
Менеджер: /revenue
Бот: Выберите дату для ввода выручки:
[Календарь с кнопками дат]
Менеджер: [Нажимает на "15.09.2023"]
Бот: Введите выручку за 15.09.2023 для магазина "Магазин №1":
Менеджер: 120500.50
Бот: ✓ Выручка 120500.5 для магазина "Магазин №1" за 15.09.2023 успешно сохранена.
\end{verbatim}
\end{tcolorbox}

\subsection{Проверка статуса выполнения плана}
\begin{enumerate}
    \item Отправьте команду \texttt{/status}
    \item Получите информацию о текущем плане магазина, фактической выручке за месяц и проценте выполнения плана.
\end{enumerate}

\begin{tcolorbox}[colback=blue!5, title=Пример просмотра статуса]
\begin{verbatim}
Менеджер: /status
Бот: 📊 Статус выполнения плана для магазина "Магазин №1":

План на месяц: 300000
Текущая выручка: 187450
Процент выполнения: 62.5%
\end{verbatim}
\end{tcolorbox}

\section{Руководство для администраторов}

Администраторы имеют полный доступ к системе, включая управление магазинами, менеджерами и отчетностью.

\subsection{Основное меню}
После авторизации вам будет доступно меню с командами:
\begin{itemize}
    \item \texttt{/report} — генерация отчетов с визуализацией
    \item \texttt{/setplan} — установка плана для магазина
    \item \texttt{/assign} — привязка менеджера к магазину
    \item \texttt{/addstore} — добавление нового магазина
    \item \texttt{/addmanager} — добавление нового менеджера
    \item \texttt{/users} — список пользователей системы
    \item \texttt{/stores} — список магазинов с планами и привязанными менеджерами
    \item \texttt{/help} — вызов справки с командами
\end{itemize}

\subsection{Генерация отчетов}
\begin{enumerate}
    \item Отправьте команду \texttt{/report}
    \item Дождитесь генерации отчетов. Система подготовит:
    \begin{itemize}
        \item Excel файл с подробной информацией о выручке
        \item Визуальные отчеты в виде матрешек с показателями выполнения плана
    \end{itemize}
\end{enumerate}

\begin{tcolorbox}[colback=green!5, title=Информация о визуальных отчетах]
Каждая матрешка представляет собой магазин со следующей информацией:
\begin{itemize}
    \item Процент заполнения матрешки соответствует проценту выполнения плана
    \item Все матрешки имеют единый стиль стального синего цвета
    \item Слева от матрешки отображается числовое значение процента выполнения
    \item Справа от матрешки выводится название магазина, последняя сумма выручки с датой и общая информация о выполнении плана
\end{itemize}
\end{tcolorbox}

\subsection{Установка плана продаж}
\begin{enumerate}
    \item Отправьте команду \texttt{/setplan}
    \item Выберите магазин из списка
    \item Введите новое значение плана (число)
\end{enumerate}

\begin{tcolorbox}[colback=green!5, title=Пример установки плана]
\begin{verbatim}
Администратор: /setplan
Бот: Выберите магазин для установки плана:
[Список магазинов]
Администратор: [Выбирает "Магазин №1"]
Бот: Введите новый план числом:
Администратор: 300000
Бот: ✓ План магазина "Магазин №1" обновлён до 300000.
\end{verbatim}
\end{tcolorbox}

\subsection{Добавление нового магазина}
\begin{enumerate}
    \item Отправьте команду \texttt{/addstore}
    \item Введите название нового магазина
    \item Введите плановое значение выручки для магазина
\end{enumerate}

\begin{tcolorbox}[colback=green!5, title=Пример добавления магазина]
\begin{verbatim}
Администратор: /addstore
Бот: Введите название нового магазина:
Администратор: Магазин на Невском
Бот: Введите план для магазина (число):
Администратор: 250000
Бот: ✓ Новый магазин 'Магазин на Невском' успешно добавлен с планом 250000.
\end{verbatim}
\end{tcolorbox}

\subsection{Добавление нового менеджера}
\begin{enumerate}
    \item Отправьте команду \texttt{/addmanager}
    \item Введите имя нового менеджера
    \item Введите фамилию нового менеджера
    \item Выберите магазин для привязки или опцию "Без привязки"
\end{enumerate}

\begin{tcolorbox}[colback=green!5, title=Пример добавления менеджера]
\begin{verbatim}
Администратор: /addmanager
Бот: Введите имя нового менеджера:
Администратор: Иван
Бот: Введите фамилию менеджера:
Администратор: Петров
Бот: Выберите магазин для привязки менеджера (или выберите 'Без привязки'):
[Список магазинов и опция "Без привязки"]
Администратор: [Выбирает "Магазин №2"]
Бот: ✓ Менеджер Иван Петров успешно добавлен, привязан к магазину: Магазин №2.
\end{verbatim}
\end{tcolorbox}

\subsection{Привязка менеджера к магазину}
\begin{enumerate}
    \item Отправьте команду \texttt{/assign}
    \item Выберите менеджера из списка существующих менеджеров
    \item Выберите магазин для привязки менеджера
\end{enumerate}

\begin{tcolorbox}[colback=green!5, title=Пример привязки менеджера]
\begin{verbatim}
Администратор: /assign
Бот: Выберите менеджера для привязки к магазину:
[Список менеджеров]
Администратор: [Выбирает "Иван Петров"]
Бот: Выберите магазин для привязки:
[Список магазинов]
Администратор: [Выбирает "Магазин №3"]
Бот: ✓ Менеджер Иван Петров успешно привязан к магазину Магазин №3.
\end{verbatim}
\end{tcolorbox}

\subsection{Редактирование существующего магазина}
\begin{enumerate}
    \item Отправьте команду \texttt{/editstore}
    \item Выберите магазин из списка
    \item Выберите, что хотите изменить: название или план
    \item Введите новое значение
\end{enumerate}

\begin{tcolorbox}[colback=green!5, title=Пример редактирования магазина]
\begin{verbatim}
Администратор: /editstore
Бот: Выберите магазин для редактирования:
[Список магазинов]
Администратор: [Выбирает "Магазин №1"]
Бот: Выберите, что хотите изменить для магазина 'Магазин №1':
[Варианты: "Изменить название", "Изменить план"]
Администратор: [Выбирает "Изменить название"]
Бот: Введите новое название магазина:
Администратор: Центральный магазин
Бот: ✓ Название магазина успешно изменено на 'Центральный магазин'.
\end{verbatim}
\end{tcolorbox}

\subsection{Редактирование менеджера}
\begin{enumerate}
    \item Отправьте команду \texttt{/editmanager}
    \item Выберите менеджера из списка
    \item Выберите, что хотите изменить: имя, фамилию или привязку к магазину
    \item Введите новое значение
\end{enumerate}

\begin{tcolorbox}[colback=green!5, title=Пример редактирования менеджера]
\begin{verbatim}
Администратор: /editmanager
Бот: Выберите менеджера для редактирования:
[Список менеджеров]
Администратор: [Выбирает "Иван Петров"]
Бот: Выберите, что хотите изменить для менеджера 'Иван Петров':
[Варианты: "Изменить имя", "Изменить фамилию", "Изменить магазин"]
Администратор: [Выбирает "Изменить магазин"]
Бот: Выберите новый магазин для менеджера (или 'Без привязки'):
[Список магазинов и опция "Без привязки"]
Администратор: [Выбирает "Центральный магазин"]
Бот: ✓ Менеджер 'Иван Петров' теперь привязан к магазину 'Центральный магазин'.
\end{verbatim}
\end{tcolorbox}

\subsection{Просмотр списка пользователей}
\begin{enumerate}
    \item Отправьте команду \texttt{/users}
    \item Получите список всех пользователей с указанием их ролей и привязок к магазинам
\end{enumerate}

\subsection{Просмотр списка магазинов}
\begin{enumerate}
    \item Отправьте команду \texttt{/stores}
    \item Получите список всех магазинов с указанием планов и привязанных менеджеров
\end{enumerate}

\section{Часто задаваемые вопросы}

\subsection{Как обновить данные о выручке?}
Если вы хотите изменить данные о выручке за определенный день, просто повторно введите команду \texttt{/revenue}, выберите ту же дату и введите новое значение. Система автоматически обновит существующую запись.

\subsection{Могут ли несколько менеджеров быть привязаны к одному магазину?}
Да, к одному магазину может быть привязано несколько менеджеров. Каждый из них сможет вводить данные о выручке для этого магазина.

\subsection{Что означают цвета матрешек в отчете?}
Цвет заполнения матрешки отражает процент выполнения плана:
\begin{itemize}
    \item Красный — низкое выполнение плана (менее 30\%)
    \item Желтый — среднее выполнение плана (30-70\%)
    \item Зеленый — высокое выполнение плана (более 70\%)
\end{itemize}

\subsection{Могу ли я вводить выручку за будущую дату?}
Нет, система не позволяет вводить данные о выручке для будущих дат. Вы можете выбрать только текущую дату или даты из прошлого.

\section{Технические требования}

Для использования бота StoreKeeper требуется:
\begin{itemize}
    \item Учетная запись в Telegram
    \item Подключение к интернету
    \item Добавление бота StoreKeeper в контакты
\end{itemize}

\section{Обратная связь и поддержка}

Если у вас возникли вопросы или предложения по улучшению функциональности бота, пожалуйста, свяжитесь с администратором системы.

\end{document}
